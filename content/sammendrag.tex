\chapter{Sammendrag}

Automatisk orddeling er en viktig del av dagens systemer for tekstsetting.  Dessverre skjer ikke dette uten feil. Systemene tilbyr heller ingen kontroll over hvilke typer regler som skal benyttes ved orddeling. I det norske språk har vi flere regler som kan gi opphav til delepunkt i et ord. Det kan argumenteres for at disse delepunktene er av forskjellig kvalitet, og det er derfor ønskelig å kunne ha en større kontroll over dette når en skal sette en tekst. I denne oppgaven ser vi på muligheten for implementasjon av en regelbasert tilnærming til orddelingsproblemet, som skal gi økt presisjon og økt kontroll over orddelingen.