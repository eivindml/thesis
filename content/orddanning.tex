\chapter{Orddanning}
\label{sec:orddanning}

For å få en god forståelse for de norske reglene for orddeling er det nødvendig med en forståelse for orddanning (hvordan ord bygges opp) og noe om ordklasser (grupper vi deler ordene inn i). Norsk Referansegrammatikk \cite{faarlund1997norsk} gir en god innføring i disse temaene og jeg vil gjengi noe av det innholdet her. I følge Wikipedia \cite{wiki-nrg} er denne grammatikken «i praksis det sentrale referanseverket når det gjeld norsk grammatikk».

\begin{center}
{\huge\color{gray!50}{\decofourleft}}
\end{center}

I norsk kan vi ha \term{komplekse ord} og \term{enkle ord} -- ord som ikke kan deles opp i mindre deler. Disse ordene kaller vi for \term{rotord}. 

\ex{hus, hage, båt}

Av disse rotordene har vi muligheter til å danne nye ord gjennom \textit{bøyning}, \textit{avledning}, \textit{sammensetning}; eller kombinasjoner av disse (se figur \ref{fig:ordtre}). 
\marginelement[-7]{
\tikzset{
  treenode/.style = {align=center, inner sep=0pt, text centered},
  arn_n/.style = {treenode},% arbre rouge noir, noeud noir
  arn_r/.style = {treenode, red},% arbre rouge noir, noeud rouge
}
\begin{tikzpicture}[->,>=stealth',level/.style={sibling distance = 2.4cm/#1,
  level distance = 1.1cm,   text height=1.5ex,
    text depth=.25ex},font=\footnotesize\sffamily] 
\node [arn_r] {umuligheten}
    child{ node [arn_n] {stamme} 
            child{ node [arn_n] {prefiks} 
							child{ node (1) [arn_r] {u}}
            }
            child{ node [arn_n] {rot}
							child{ node (2) [arn_r] {mulig}}
            } 
            child{ node [arn_n] {avledning}
							child{ node (3) [arn_r] {het}}
            } 
    }
    child{ node [arn_n] {bøyningsendelse}
            child{ node [arn_r, below=2.7em] {en} }
		}
; 
\end{tikzpicture}
    \captionof{figure}[Eksempel på orddanning]{Trestruktur som viser hvordan ord dannes gjennom avledningsaffikser og bøyningsendelser.}
    \label{fig:ordtre}
}
Når vi kombinerer disse til å danne nye ord gjøres det trinnvis i en ikke-tilfeldig rekkefølge, da rekkefølgen vil ha betydning på det nye ordet som dannes. Hvis vi sammenligner ordene rødvinsglass og krystallvinglass ser vi betydningen av dette. Vi antar at rødvinsglass er satt sammen av rødvin og glass, som betyr «glass for rødvin», mens derimot krystallvinglass er satt sammen av ordene krystall og vinglass som betyr «vinglass av krystall» (se figur \ref{fig:rodvin} og \ref{fig:krystall}).

\marginelement[-0.5]{
\tikzset{
  treenode/.style = {align=center, inner sep=0pt, text centered},
  arn_n/.style = {treenode},% arbre rouge noir, noeud noir
  arn_r/.style = {treenode, red},% arbre rouge noir, noeud rouge
}
\begin{tikzpicture}[->,>=stealth',level/.style={sibling distance = 2.4cm/#1,
  level distance = 1.1cm,   text height=1.5ex,
    text depth=.25ex},font=\footnotesize\sffamily] 
\node [arn_r] {rødvinsglass}
    child{ node [arn_r] {rødvins} 
            child{ node [arn_r] {rød} 
            }
            child{ node (1) [arn_r] {vins}
            } 
    }
    child{ node [arn_r, right of = 1,xshift=1cm] {glass}
		}
; 
\end{tikzpicture}
    \captionof{figure}[Syntakstre for «rødvinsglass»]{Syntakstre som viser hvordan ordet «rødvinsglass» er satt sammen.}
    \label{fig:rodvin}
}

\marginelement[9]{
\tikzset{
  treenode/.style = {align=center, inner sep=0pt, text centered},
  arn_n/.style = {treenode},% arbre rouge noir, noeud noir
  arn_r/.style = {treenode, red},% arbre rouge noir, noeud rouge
}
\begin{tikzpicture}[->,>=stealth',level/.style={sibling distance = 2.4cm/#1,
  level distance = 1.1cm,   text height=1.5ex,
    text depth=.25ex},font=\footnotesize\sffamily] 
\node [arn_r] {krystallvinglass}
    child{ node [arn_r, left of = 1,xshift=0.5cm] {krystall}
		}
    child{ node [arn_r] {vinglass} 
            child{ node (1) [arn_r] {vin} 
            }
            child{ node [arn_r] {glass}
            } 
    }
; 
\end{tikzpicture}
    \captionof{figure}[Syntakstre for «krystallvinglass»]{Syntakstre som viser hvordan ordet «krystallvinglass» er satt sammen.}
    \label{fig:krystall}
}

Videre i denne teksten vil jeg først beskrive de relevante ordklassene, før vi ser på orddanning gjennom bøyninger, avledninger og sammensetninger.

\clearpage
\section{Ordklasser}
\label{sec:ordklasser}

\term{Leksemer}\sidenote{I følge Wikipedia er et leksem «et abstrakt begrep innenfor språkvitenskapen som viser til alle ord som er forskjellige former av et bestemt ord.» \cite{wiki-leksem} Eksempelvis er bil, bilen, bilene og biler ett leksem.} har ulik form og funksjon, som gjør at vi kan dele dem inn i forskjellige \term{ordklasser}. Ordklassene gjør det mulig å kunne beskrive grupper av ord på en generell basis, for igjen å kunne gi gramamtiske regler basert på dette. Tre kriterier ligger til grunne for å danne grunnlaget for ordlassene: morfologiske (går på hvilke affiks ordene kan få, spesielt bøyningsendelsen), syntaktiske (funksjonen ordet har i setningen) og semantiske (skiller blant annet mellom leksikalse ord, grammatiske ord og pro-ord). Norsk referansegrammatik som ligger til grunne for denne teksten har en ordklasseinndeling som avviker noe fra den tradisjonelle inndelingen\cite{faarlund1997norsk}:

\begin{quote}
… den tradisjonelle inndelinga er basert på ulike kriterer og skaper ofte problemer i språkbeskrivelsen. Her skal vi sette opp en ordklasseinndeling ut fra helhetlige inndelingskriterier, men likevel slik at den ikke bryter mer enn nødvendig med den tradisjonelle inndelinga.
\end{quote}

I Norsk referansegramatikk benyttes de morfologiske, altså bøyningen,  som kriterie for ordklasseinndelingen. Ved andre nivå benyttes syntaktiske kriterier. I denne oppgaven er jeg kun interessert i ordklassene som legger morfologiske kriterier til grunne, altså \term{ord med bøyning}. Det er ordklassene substantiv, verb, adjektiver, pronomen og determinativ.  Av disse er vi i oppgavens sammenheng kun interessert i ord(under)klasser som har den syntaktisk egenskap \textit{leksikalske}. Da sitter vi i gjen med ordklassene \term{substantiv} (utelatt proprium som står ubøyd), \term{verb} og \term{adjektiv} (utelatt ordenstall som står ubøyd).

\subsection{Ordklasseinndeling}

\paragraph{Substantiv} Typisk navn og betegnelser på gjenstander, idivider og abstrakte begreper.
	\begin{itemize}
		\item Morfologisk kriterium: Bestemt artikkel.
		
		\ex{\textit{bil} -- \textit{bilen}}{}
		
		\item De fleste har også tallbøning.
		
		\ex{\textit{bil} -- \textit{biler}}{}
		
	\end{itemize}
\paragraph{Verb} Betegner ofte en handling, prosess eller tilstand.
	\begin{itemize}
		\item Morfologisk kriterium: Tempusbøyning.

		\ex{\textit{kaster} -- \textit{kastet}}{}
		
		\item I tillegg til finitt (presens og preteritum) har verb også infinitt (infitiv og partisipp). 
	\end{itemize}
	
\paragraph{Adjektiv} Først og fremst ord som kan gradbøyes, altså får lagt til (e)ere eller (e)st, med eller uten vokalveksling.
	\begin{itemize}
		\item Morfologisk kriterium: Gradbøyning.
		
		\ex{\textit{fin} -- \textit{finere} -- \textit{finest}}{}\newline
		\exx{\textit{stor} -- \textit{større} -- \textit{størst}(vokalveksling)}{}
		
		\item Bøyes også i genus, tall og bestemthet.
	\end{itemize}

\section{Avledninger}

En \term{avledning}
\sidetable[Liste over avledningsprefikser]{%
	Avledningsprefikser hentet fra Norsk referansegrammatikk \cite{wiki-nrg}. Tegnforklaring: \textbf{N} (Norske), \textbf{G} (Germanske) og \textbf{F} (Fremmedspråklige). Hentet fra Norsk Referansegrammatikk \cite{wiki-nrg}.
  \label{table:prefikser}}{%

\begin{tabularx}{\marginparwidth}{lp{0.7\marginparwidth}}
\toprule
\multicolumn{2}{c}{\textbf{Prefikser}}                                                                                                                                                                                                                                                                                                                                                                                                                                \\ \midrule
\multicolumn{1}{l}{\textbf{N}}    & \multicolumn{1}{l}{u, mis, van, be, for, fore, føre}                                                                                                                                                                                                                                                                                                                                                           \\ 
\multicolumn{1}{l}{\textbf{G}} & \multicolumn{1}{l}{an, bi, er, ge, unn}                                                                                                                                                                                                                                                                                                                                                                        \\
\textbf{F}                & a, ad, an, andro, ante, anti, antro, bi, bio, centi, de, des, di, dia, dis, dys, eks, erke, eu, ev, geo, giga, hetero, homo, hyper, hypo, il, im, in, inter, intra, iso, ko, kon, kontra, kvasi, makro, maksi, mega, meta, midi, mikro, milli, mini, mono, multi, non, orto, pan, para, poly, post, pre, pro, proto, pseudo, psevdo, re, retro, semi, sub, super, syn, tele, trans, ultra, uni, vara, vise, øko \\ \bottomrule
\end{tabularx}
}
 er ord som dannes ved hjelp av avledningsaffikser\footnote{Et affiks er en orddel som føyes til en rot eller stamme. Kan gi videre avledninger og bøyninger, og deles videre inn i \textit{prefiks}, \textit{innfiks} og \textit{suffiks}.} enten som prefikser hvor et affiks tilføres foran ordstammen (se tabell \ref{table:prefikser} for liste over ofte brukte prefikser).

\ex{\textit{mis}forstå}


Eller som et suffiks hvor affikset tilføres enden av ordstammen (se tabell \ref{table:suffikser} for ofte brukte suffikser).

\ex{kjær\textit{lig}}

\sidetable[Liste over norske avledningssuffikser]{%
	Norske avledningssuffikser hentet fra Norsk referansegrammatikk \cite{wiki-nrg}.
  \label{table:suffikser}}{%

\begin{tabularx}{\marginparwidth}{p{0.9\marginparwidth}}
\toprule
\multicolumn{1}{c}{\textbf{Suffikser}}                                                                                                                                                                                                                                                                                                                                                                                                                                \\ \midrule
ing, ning, ling, else, sel, nad, sjon, er, ar, dom, skap, het, itet \\ \bottomrule
\end{tabularx}
}

Avledninger i norsk språk er det vi vil kalle for \term{produktive}. Måten vi danner ord på er produktiv når nye ord kan dannes på samme måte. For eksempel kan vi alltids lage substantiver av verb ved å påføre suffikset \textit{-ing}. Slik kan vi alltid lage nye ord av verb som kommer inn i det norske språket.

\ex{skuldersurf\textit{ing}}

Vi kan også danne ord ved avledning ved å tilføre avledningssuffikset -het til adjektiv og danne nye substantiv.

\ex{god\textit{het}}

\section{Bøyninger}

Når et ord bøyes får det et tillagt betydningselement, stort sett ved tillegg av bøyningsendelser (suffiks). Eksempelvis \textit{båt} -- \textit{båter}. Eller ved såkalt indre bøyning, også kalt vokalskifte. Eksempelvis \textit{mann} -- \textit{menn}. Norsk referansegrammtikk \cite{faarlund1997norsk} nevner åtte forskjellige bøyningskategorier: \textit{genus} (kjønn), \textit{tall} (numerus), \textit{bestemthet}, \textit{kasus}, \textit{grad}, \textit{tempus}, \textit{modus} og \textit{diatese}. I denne sammenhengen her er vi kun avhengig av bøyningskategoriene som gjelder for ordklassene substantiv, verb, og adjektiv --- alltså genus, bestemthet, tall, tempus, og grad (se kapittel \ref{sec:ordklasser} for beskrivelse av ordklassene).


\paragraph{Genus} Tre vanlige kjønn, ord fra visse ordklasser bøyes forskjellige etter hvilke ordklasser de hører til: \term{maskulin} (hannkjønn), \term{feminum} (hunnkjønn) eller \term{nøytrum} (intentkjønn).

\paragraph{Tall} Uttrykker tallforhold gjennom bøyninger og bøyes i entall og flertall.

\ex{båt}{(entall)}\newline
\exx{båt\textit{er}}{(flertall)}\newline

\paragraph{Bestemthet} Skiller mellom enheter som er spesifikke og identifiserbare (bestemt) eller de som ikke er det (ubestemt) \cite{wiki-bestemthet}. Kan uttrykkes på flere måter, spesielt viktig her, etterhengt \term{bestemt artikkel}:

\ex{gutt}{(ubestemt)}\newline
\exx{gutt\textit{en}}{(bestemt)}\newline

Bestemt artikkel endrer seg i genus og tall etter ordet.

\ex{båt\textit{en}}{(mask. ent.)}\newline
\exx{båt\textit{ene}}{(best. flert.)}\newline

\paragraph{Grad} En sammenlignen, komperasjon, av subjekter. Bøyes i tre grader: \textit{positiv}, \textit{komperativ} og \textit{superlativ}.

\ex{fin}{(grunform)}\newline
\exx{fin\textit{ere}}{(komperativ)}\newline
\exx{fin\textit{est}}{(superlativ)}

\paragraph{Tempus} Angir tidspunkt for handlingen eller tilstanden som vises til, og bøyes i \term{presens} og \term{preteritum}.

\ex{hør\textit{er}}{(presens)}\newline
\exx{hør\textit{te}}{(preteritum)}\newline

\section{Ordsammensettning}

En ordsammensetning er en sammenstilling av to ord for å danne et nytt. Disse kan være mer eller mindre \term{gjennomskuelige} eller mer eller mindre \term{leksikaliserte}. Gjennomskuelige sammensetninger er ord vi intutivt kan forstå betydningen av uavhengig om vi har sett dem tidligere. De leksikaliserte sammensetningene er orddanninger vi ikke forstår intuitivt som gjerne er konstruert mer metaforiske.

\ex{vinter+dag}{(gjennomskuelig)}\newline
\exx{danse+løve}{(leksikalisert)}

En sammensetning er satt sammen av ord som alle bør kunne opptre på egenhånd som egne ord. Dermed vil \texttt{eple+kake} være et sammensatt ord, men ikke \textit{pult+ost}. Selv om «pult» kan ha flere betydninger i norsk, er det ingen av disse som refereres til i denne sammenhengen\sidenote{I følge Norsk etymologisk ordbok kommer ordet «pult» fra middelalderlatinordet «pulta» som betyr grøt eller velling. \cite{de2013norsk} Men ikke et ord som står oppført i ordboka alene med denne definisjonen.}. Det er altså ikke et ord som har egen betydning når det står alene. 

En sammensetning vil bestå av først et forledd, som oftest står ubøyd, og så et etterledd, som stort sett bestemmer ordklassen.

\ex{kryp+inn}{(=subst., inn=prep.)}

Forleddet i sammensetningen kan igjen være sammensatt av flere ledd. Det kan altså bestå av flere rotord.

\ex{[flyve+maskin]+instruktør}

Punktet der forleddet og etterleddet treffer hverandre kalles for \term{sammensetningsfugen}. Den vanligste formen for fuge er en nullfuge, der forledd og etterledd settes sammen uendret. Men i en del tilfeller oppstår det fuge-s eller fuge-e (og i mer sjeldene tilfeller andre bindebokstaver eller bindeord), som binder leddene sammen. 

\section{Bindebokstaver}
\label{sec:fuge-bokstav}

Det finnes flere typer bindeelementer som binder sammen forledd og etterledd og som tydligere kan vise hva som er hovedgrensen i ordet. Vanligste bindingen er nullfuge, hvor begge ledd har samme form som grunnform.

\ex{båt+hus}{(nullfuge)}

Ifølge Munthe (1972, i Johannesen og Hauglins artikkel «An automatic analysis of Norwegian compounds» \cite{johannessen1996automatic}) er 10,4 \% av alle ord i norske tekster sammensatte. Av disse har omtrent 75 \% av sammensatte ord nullfuge; hvor leddene er satt sammen uten en bindebokstav \cite{johannessen1996automatic}. Det betyr at rundt 25 \% av alle sammensatte ord i norsk inneholder en bindebokstav. Av bindebokstaver har vi de vanligste: \textit{-s} og \textit{-e}. Noe mer sjeldent ser vi: \textit{-en}, \textit{-a}, \textit{-er} og \textit{-es}. I svært få tilfeller kan det oppstå: \textit{-o} og \textit{-ium → -ie} \cite{faarlund1997norsk,bindebokstaver}.

\ex{vindu\textit{s}vask}{(-s)}\newline
\exx{jule\textit{e}kveld}{(-e)}\newline
\exx{hit\textit{en}for}{(-en)}\newline
\exx{ås\textit{a}tro}{(-a)}\newline
\exx{student\textit{er}samfunn}{(-er)}\newline
\exx{makt\textit{es}løs}{(-es)}\newline
\exx{afr\textit{o}amerikansk}{(-o)}\newline
\exx{labrator\textit{ie}forsøk}{(-ium → -ie)}\newline

\begin{center}
{\huge\color{gray!50}{\decofourleft}}
\end{center}

Norsk referansegramatikk nevner at binde-s og binde-e er de desidert mest frekvente av bindebokstavene (se kapittel~\ref{sec:fuge-frekvens} for en ytterligere analyse av frekvensen). Derfor vil jeg gi en nærmere beskrivelse av spesielt disse to i de følgende kapitlene. 

\subsection{Binde-s}
\label{sec:ord-bind1}

Det finnes ingen klare regler for binde-s og når den skal tilføres  sammensetningsfugen, men det finnes noen rettledene punkter. Det er en viss taleforskjell å spore. Ved sammensetninger uten binde-s og et forledd som er et enstavelsessubstantiv ser vi noramlt får tonem\sidenote{Tonem er en betydningsdifferensierende tonegang («musikk») ved uttale av ord som gjør at vi kan skille dem. Det er mange ord som kun kan skilles gjennom tonemet, ord som er homografe, men som ikke er homonyme -- for eksempel ordet snekker. Vi har \textit{(en) snekker} (håndtverkeren) og \textit{(flere) snekker} (båten), hvor de henholdsvis har tonem én og tonem to.} to (\textit{bønner} -- erteblomsten). Sammensetninger med binde-s uttales helst med tonem én (\textit{bønner} -- religiøs praksis) når forleddet er enstavet. Når forleddet er to eller flerstavet får sammensetningen samme tonem som forleddet.

Norsk referansegrammatikk \cite{faarlund1997norsk} lister også opp en rekke andre karakteristikker ved binde-s, som jeg vil gjengi her.

\begin{items}
	\item Forleddet er et rotord …
	\begin{itemize}
		\item … og forleddet ender på \textit{-s}, \textit{-sj} eller en konsonantgruppe som har lik lyd, gir ingen binde-s.
		
		\ex{løs+katt}{}
		
		\item … og forleddet ender på en vokal, gir sjeldent binde-s, spesielt når forleddet er enstavet.
		
		\ex{le+skur}{}
		
		\item … og forleddet ender på konsonant, gir binde-s i en god del tilfeller, men det er fortsatt vanligere uten binde-s. Binde-s er spesielt sjeldent etter trykklett \textit{en}, \textit{el}, \textit{er} eller ord på \textit{ft}, \textit{kt} og \textit{m}.
		
		\ex{vakt+avløsning}{}
		
	\end{itemize}
	
	\item Når forleddet er en avledning …
	\begin{itemize}
		\item … og forleddet er prefiksavledet, gir en tendens til binde-s, spesielt ved prefiksene \textit{an}, \textit{be}, \textit{bi} og \textit{for(e)}.
		
			\ex{forslag\textit{s}+kasse}{}
		
		\item … og forleddet er suffiksavledet, gjør det vanlig med binde-s, spesielt etter nordiske suffikser som \textit{(n)ing}, \textit{dom}, \textit{else}, \textit{het}, \textit{leik}, \textit{nad}, \textit{sel} eller \textit{skap}.
		
		\ex{anretning\textit{s}+rom}{}
		
		\item … og forleddet er suffiksavledet av fremmed suffiks, gjør binde-s ganske vanlig. 
		
		\ex{plenum\textit{s}+forelesning}
		
		\item … og forleddet er et verbalsubstantiv (uten suffiks), gjør det vanlig med binde-s. 
		
		\ex{hørsel\textit{s}+vern}{}
	\end{itemize}
	
	\item Når forleddet er en sammensetning …
	\begin{itemize}
		\item … gir det i større grad binde-s enn når forleddet er et rotord.
	
		\ex{vin+glass}{}\newline
		\exx{rødvin\textit{s}+glass}{}
		
		\item … og ender på vokal gir det sjeldent binde-s.
		
		\ex{bispedømme+råd}{}
		
		\item … og forleddet ender på \textit{en}, \textit{el} eller \textit{er} får man normalt ikke binde-s. Untaktet er de som ender på \textit{-sel}.
		
		\ex{næringsmiddel+industri}{}\newline
		\exx{anførsel\textit{s}+tegn}{}
		
	\end{itemize}
\end{items}

\subsection{Binde-e}
\label{sec:ord-bind2}

Binde-e forekommer også relativt ofte, men ikke like hyppig som fuge-s. Det finnes heller ingen veldig klare regler for binde-e, men vi har igjen noen karakteristikker \cite{faarlund1997norsk}.

\begin{items}
	\item Binde-e forekommer ofte når forleddet er et enstavet substantiv som ender på konsonant. 
		
		\ex{jule+kake}{}
		
	\item Binde-e forekommer sjeldnere foran etterledd på vokal.

	 \ex{jul+aften (men jule+kveld)}{}
	 
	\item Når forleddet er sammensetning, er binde-e sjelden. 
	 
	\ex{storsild+fiske (men sild\textit{e}+fiske)}{}

	\item Ved forledd som er substantiv og ender på trykklett \textit{-e}, hører \textit{e}-en med til forleddet og regnes ikke som en binde-e. 

	\ex{jente+barn}{}	
\end{items}


\subsection{Regler for bindebokstaver}
\label{sec:reg-bind}

Bruk av bindebokstaver er vanskelig. Det finnes ingen absolutt regelbruk for dette. Norsk referansegrammatikk \cite{faarlund1997norsk} gir noen regler for hvordan disse brukes og opptrer i det norske språk, men er noe ufullstendig. Janne Bondi Johannessen og Helge Hauglin gir noen ytterligere rettningslinjer for analyse av binde-s og binde-e i sammensatte ord i norsk språk. Disse retningslinjene vil være til stor nytte i algoritmen for å dele ord, og jeg vil raskt gjengi disse her\cite{johannessen1996automatic}:

\begin{enum}
	\item Tolk sammensetningen som en kombinasjon av to rotord, uten fuge om mulig.
	
		\ex{løve-manke}{}\newline
		\exx{Ikke: løv-e-manke}{}
	
	\item Tolkning som binde-s er foretrukket om det er en tvetydig tolkning der s-en også kan ingå som første bokstav i et verbalt etterledd.
	
		\ex{aluminium-s-nakke}{}\newline
		\exx{Ikke: aluminium-snakke}{}
	
	\item Tolkning som binde-s er foretrukket fremfor nullfuge om forleddet i seg selv er et sammensatt ord.
	
		\ex{lesesal-s-turer}{}\newline
		\exx{Ikke: lesesal-sturer}{}
	
	\item Binde-s kan aldri følge binde-e og visa versa.
		
		\ex{hest-e-sal}{}\newline
		\exx{Ikke: hest-e-s-al}{}
	
	\item Ved to analyser som gir likt antall medlemmer og ingen bindebokstav er involvert, velg, hvis mulig, analysen som er et substantiv.
		
		\ex{hun-dyr}{(S)}\newline
		\exx{Ikke: hund-yr}{(V)}
	
	\item Ved to like analyser med tanke på bindebokstav og regelen over, og en av dem har et forledd som i seg selv er et sammensatt ord, velg den.
		
		\ex{fagplan-arbeid}{}\newline
		\exx{Ikke: fag-planarbeid}{}
	
	\item Binde-e kan kun settes sammen med en stamme som har enkelt stavelse.
		
		\ex{hest-e-ekvipasje}{}\newline
		\exx{tre-hest-ekvipasje}{}\newline
		\exx{Ikke: tre-hest-e-ekvipasje}{}
	
	\item Stammer kan komme før forleddet før -e så lenge de ikke danner sammensetning med forleddet.
		
		\ex{konge-hus-hest-e-ekvipasje}{}\newline
		\exx{Ikke: konge-hus-hest-ekvipasje}{}
	
	\item Binde-s opptrer ikke etter en konsonantsekvens med sibilanter … 
		
		\ex{busk-spilling}{}\newline
		\exx{Ikke: busk-s-pilling}{}
	
	\item … med mindre forleddet er sammensatt.
		
		\ex{enebærbusk-spilling}{}\newline
		\exx{enebærbusk-s-pilling}{}
	
	\item Hvis forleddet er ukjent, velg analysen med det lengste etterleddet.
		
		\ex{Ibsen-stykket}{}\newline
		\exx{Ikke: Ibsens-tykke}{}
	
\end{enum}

