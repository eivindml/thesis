\chapter{Orddeling}
\label{sec:orddeling}

TODO: Påpek eksempel med orangutang/alene-feil i Skriveregler. Velger å tolke annerledes en Gunnar.
Nevne endring i regel for deling av navn. Før frabedt, nå tillat: Kilde skriveregler.

Orddeling ved linjeskift i det norske språk er regelbasert (i motsats til for eksempel amerikansk-engelsk som er uttalebasert). Reglene for orddeling er en del av det som kalles tekniske skriveregler, som er utviklet over flere år fra mange kilder. Forskjellige komitéarbeid hvor blant annet språkprofessor Finn-Erik Vinje har vært delaktig, standardiseringsarbeid og annet.\cite{Simonsen2015} Disse reglene ble gjort gjeldene i 1973 og først publisert i Finn-Erik Vinjes bok Skriveregler. Det er Språkrådet som er ansvaret for å sette rettningslinjer og klargjøre hva som er korrekte gjeldende regler for orddeling. Dette mandatet fikk de etter Stortingsmelding nr. 100. Språkrådet har i paragraf 3 av sine vedtekter fått følgende rolle:

\begin{quote}
Språkrådet gir råd og rettleiing om tekniske skrivereglar og skal så langt det er tenleg, klargjera kva reglar som er obligatoriske for korrekt språk.
\end{quote}

\section{Reglene}
\label{sec:orddelingsreglene}

Norsk rettskrivning benytter seg av to hovedregler for orddeling, \textit{ordleddsregelen} og \textit{enkonsonantregelen}, samt kravet om at det alltid skal være minst én vokal per linje. Disse er enkle å forstå, men problemene oppstår ved alle unntakene og tvedydighetene man må forholde seg til. Reglene kan kort oppsummeres slik:

\begin{description}
	\item[Ordleddsregelen] Del ord mellom betydningsbærendebærende og/eller lett gjenkjennelige orddeler.
	\item[Enkonsonantregelen] Uavhengig av ordets betydning, la én konsonant følge med til neste linje.
\end{description}

I mange tilfeller kan man velge fritt mellom reglene. Det viktigste prinsippet er å ungå komiske, forvirrende eller meningsendrende ordbilder. Videre i teksten vil jeg forklare reglene og deres unntak i mer detalj. Innholdet i denne teksten er basert på skriveriene til Finn-Erik Vinje i boken Skriveregler \cite{vinje}.

\subsubsection{Ord og uttrykk som ikke skal deles}

\begin{inparaenum}[\itshape a\upshape)]
\item Enstavelsesord, 
\item forkortelser,
\item  årstall, datoer og andre sammenhengende siffer- eller tegngrupper,
\item og forbindelser av ordenstall i numerisk form og tilhørende substantiv,
\end{inparaenum} skal ikke deles.

	\ex{kai, dag, skjær}{(a)}\newline
	\exx{ADHD}{(b)}\newline
	\exx{1988, 299 792 458 m/s}{(c)}\newline
	\exx{14. september}{(d)}

\subsection{Unntak fra enkonsonantregelen}
\label{sec:enkons-untak}

Følgende forbindelser deles ikke når de gjengir én og samme lyd og tilhører samme stavelse:	\textit{dh, gh, gj, kj, sc, sch, sh, sj, skj, sk}.

	\ex{kan-skje}{}

\subsection{Usammensatte ord}

Ord med to eller flere stavelser, som ikke er sammensatt, eller som opfattes som ikke sammensatt, deles etter enkonsonantregelen. Det er også lov å dele mellom vokaler hvis de hører til hver sin stavelse.

	\ex{hø-re}{}\newline
	\exx{are-al}{}

\subsection{Sammensatte ord}

Sammensatte ord deles etter ordleddsreglen. Ved eventuell komposisjonsfuge tilhører fugen første leddet av sammensetningen. Ved sammensatte ord som består av flere enn to ordkomponenter så \textit{bør}\footnote{Finn-Erik Vinje bruker selv ordet bør i Skriveregler\cite{vinje}, som betyr at dette er en anbefaling, men ikke et krav!} de deles i hovedgrensen. Også ukjente og fremmede sammensetninger kan deles i sammensetningen. Hvis leddene ikke kan regnes for å være lett gjenkjennelige er det lov å dele etter enkonsonantsregelen. I enkelte tilfeller kan det være nødvendig eller ønskelig å dele sammensatte ord i ett av rotordene. Da benytter man seg av enkonsonantregelen slik den gjelder for usammensatte ord.

\ex{grad-vis}{(nullfuge)}\newline
\exx{fylkes-grense}{(binde-s til forledd)}\newline
\exx{oppskrifts-bok}{(del i hovedgrense)}\newline
\exx{atmo-sfære}{(fremmed sammensettning …)}\newline
\exx{atmos-fære}{(… men også slik)}\newline
\exx{ep-lekake}{(enkonsonantregelen)}

\subsection{Bøyninger}

Vi kan dele bøyningsformer enten etter ordleddsregelen eller enkonsonantregelen:

bøyning-ene, bøynin-gene.

Unntaket er når stammen ender på en vokal og bøyningsendelsen starter på en vokal, da bruker vi ordleddsregelen og deler før bøyningen:

sau-ene.

\subsection{Avledninger}

Avledninger benytter seg stort satt av ordleddsregelen. Avledninger med \textit{prefikser} deles etter ordleddsregelen, altså etter prefikset:

\ex{a-typisk}{}

Avledninger med \textit{suffikser} er litt mer komplisert. Om prefikset begynner på en konsonant skal det deles etter ordleddsregelen; før suffikset.

\ex{kjærlig-het}{}

Unntaket er når suffikset begynner på vokal, da kan vi dele \textit{både} med ordleddsregelen og med enkonsonantregelen.

\ex{sekter-isk}{}\newline 
\ex{sekte-risk}{}

\section{Tolkning av reglene}

TODO: Rotete kapittel og kanskje komme et annet sted?

Reglene for orddeling er ikke helt entydig definert -- det er rom for tolkninger og tvetydighet, spesielt for sammensatte ord:

\begin{items}
	\item Ved deling av sammensatte ord skal man benytte seg av ordleddsregelen. Unntaket er om «ordet ikke består av ledd som kan sies å være lett kjennelige»\cite{vinje}, da kan man dele etter enkonsonantregelen. Hvor går grensen for hva som kan sies å være lett kjennelige ordledd? Kan en person med et mer begrenset ordforråd benytte seg av mer liberal orddeling etter enkonsonantregelen?
	\item Regelen for deling av sammensatte ord sier også «I samensetninger med mer enn to ord (ledd) bør man dele der hovedgrensen går»\cite{vinje}. Dette ordet bør er vanskelig å forholde seg til. Hvor går grensen? Spesielt en datamaskin vil ha vanskeligheter med en slik beskrivelse, den trenger strengere rettningslinjer. 
	\item Til sist sier også regelen for deling av sammensatte ord «Iblant kan det imidertid være nødvendig å dele de enkelte ordene i en sammensetning ifølge reglene for enkle ord»\cite{vinje}. Igjen har vi en noe upresis formulering som gir rom for tolkning. Hva er kravene for at dette kan regnes som nødvendig? Eller kan vi benytte oss av denne regelen når vi selv ønsker det?
\end{items}

Vi ser at formuleringer som «bør», «kan sies å være lett kjennelige» og «Iblant kan det […] være nødvendig» er upresise og gir mye rom for tolkning. Vinje sier også «Ved deling av bøyningsformer […] og avledninger […] kan man i høyere grad enn ved sammensetninger velge mellom de to prinsippene»\cite{vinje}. Det betyr at det er lov til en viss grad å benytte seg av enkonsonantregelen ved sammensatte ord, men det er noe mindre valgfrihet for bruk av den. Men det er vanskelig å bli klokere på nøyaktig hvor grensen skal gå og hva som regnes for grei bruk av enkonsonantregelen. Den viktigste retningslinjen for tolkningen av dette er prinsippet om i «unngå orddelinger som gir misvisende eller komiske bilder». 

Et menneske som deler ord vil nok klare å forholde seg til disse prinsippene på en måte som gir gode resultater. Men for å kunne oppnå god og forutsigbar automatisk orddeling ved hjelp av datamaskiner skulle vi ønske at reglene var strengere og mer entydig definiert.

Kunnskapsforlagets Ordnett.no gir på sine nettsider om orddeling en strengere trinnvis definisjon for orddeling\cite{ordnett-orddeling}, hvor reglene er ordnet etter viktighet.

Først legger de til rette at grunnprinsippene om stavelsesdeling (del mellom stavelse, minst én stavelse per linje) og begynnelseskonsonanter (ikke dele ord slik at vi får to eller flere konsonanter som ikke kunne stått i begynnelsen av et ord) alltid er overordnet de syv trinnene\footnote{Finn-Erik Vinje i Skriveregler\cite{vinje} deler preteritumsbøyningsformen av nå som nå-dde, noe som strider i mot Orndett.no sitt formulerte prinsipp om å «ikke dele ord slik at vi får to eller flere konsonanter som ikke kunne stått i begynnelsen av et ord»\cite{ordnett-orddeling}. Bokmålsordboka inneholder ingen ord som starter på dobbelskrevet -d.}: 

\begin{description}
	\item[Trinn 1] Sammensetninger: Sammensatte ord deles mellom forledd og etterledd.
	\item[Trinn 2] Avledninger med prefiks (forstavelse): Avledninger med prefiks deles etter forstavelsen.
	\item[Trinn 3] Avledninger med avledningssuffiks: Avledninger med avledningssuffiks deles foran avledningsendelsen, særlig hvis den begynner med konsonant.
	\item[ Trinn 4] Én konsonant til ny linje: Ord med én konsonant mellom vokaler deles foran konsonanten. Står det to eller flere konsonanter mellom vokaler, deler vi slik at den siste konsonanten går til ny linje. 
	\item[Trinn 5] Konsonantgruppe til ny linje: Konsonanter som hører til samme stavelse, kan ikke deles.
	\item[Trinn 6] Bøyningsformer: Bøyningsformer kan […] deles etter reglene på trinn 4 og trinn 5, men de kan også deles der bøyningsendelsen begynner.
	\item[Trinn 7] Vokaler: Vi kan dele mellom vokaler som hører til hver sin stavelse.
\end{description}

En slik trinnvis og prioriert liste for orddeling er noe vi skulle ønske var definert i de offisielle retningslinjene gitt av Språkrådet. Jeg tok kontakt med Kunnskapsforlaget for å høre hvor opphavet til denne trinnvise definisjonen kommer fra eller hva grunnlaget for deres tolkning av reglene er. Som svar fikk jeg\cite{epost-orddeling}:

\begin{quote}
[…] forfatterne av Språkvett er erfarne i alle sider ved praktisk norsk og har hentet materiale fra flere kilder, ikke minst sine egne erfaringer fra korrigeringer av et stort antall av andres manus med hva som ofte deles feil.
[…] Norsk språkråd [hadde] i flere år hadde et eget utvalg som var behjelpelig med Finn-Erik Vinjes Skriveregler. Noe kan vel komme derfra.
\end{quote}

I sin masteroppgave\cite{thoresen1993virtuelle} nevner Lars Gunnar Thoresen fem hovedproblemer med norsk orddeling som han har lagt merke til. Et av disse problemene er jeg spesielt enig i og har ikke funnet noe godt svar på:

\begin{quote}
	Suffikser som bøyes. Kan suffikset \texttt{-sjon} i bøyd form \texttt{-sjonen }deles efter enkonsonantregelen slik \texttt{-sjo-n-en}? Og hva med \texttt{-else}, kan vi dele \texttt{-el-s-en}?
\end{quote}

Videre gir han fire andre problemer hvor jeg mener at to av dem ikke er problematiske:

\begin{quote}
	Sammensatte ord og enkonsonantregelen. Et eksempel er overskriften til dette avsninttet, nemlig problemord. Ordet kan deles etfter regelen om sammensatte ord problem-ord og efter enkonsonantregelen proble-mord.
\end{quote}

Vinje\cite{vinje} skriver «Det er ikke nødvendig å være konsekvent innenfor samme tekst og følge enten ordleddsregelen eller enkonsonantregelen. Orddeling tjener det formål å opprettholde en bestemt linjelengde, derfor er det praktisk […] at ord kan deles på flere måter». Videre nevner han «Før øvrig gir de to reglene ofte samme resultat» (mis-tenke, triv-sel), som hentyder at han/de er inneforstått med at det i andre tilfeller \textit{ikke} gir samme resultat.

Gunnar stiller også spørsmålet:

\begin{quote}
	Er suffikset -ing er [sic] en del av suffikset -ning slik at det kan deles som -n-ing?
\end{quote}

Norsk referansegrammatikk\cite[s.~97--98]{faarlund1997norsk} beskriver suffiksene -ing og -ning og sier «en variant av dette suffikset [-ing] er -\textit{ning}». -ning er altså en variant av -ing og det er ikke tilfelle at de er suffikser av hverandre. I tillegg er det verdt å nevne at i avledninger av verb som inneholder -n (likning / likne) er suffikset -ing og ikke -ning. Tilfellet er at det er fra disse avledningene suffikset -ning har blitt omtolket fra og spredd seg videre til verb som ikke inneholder -n (skapning). \cite[s.~98]{faarlund1997norsk}